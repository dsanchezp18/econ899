\documentclass[12pt,a4paper]{article}

% ======================== Preamble ======================== %

% Language ---------------------------------------------

\usepackage{csquotes}
\usepackage[canadian]{babel}

% Packages for title metadata ----------------------

\usepackage{authblk}

% Title metadata ---------------------------------

\title{Do fiscal incentives affect innovation? The effects of the Alberta Investor Tax Credit on patents}
\author{Daniel Sánchez-Pazmiño \\[1ex] ECON899 MA Paper}
\affil{Simon Fraser University}
\date{April 2024}

% Formatting -----------------------------------

\usepackage{setspace} % For double spacing
    \doublespacing
\usepackage[margin = 1in]{geometry} % For setting margins
\usepackage{libertinus} % Libertinus font for the whole document
\usepackage{lscape} % For landscape pages
\usepackage{hyperref} % For hyperlinks

% Misc packages -----------------------------------

\usepackage{lipsum} % For generating dummy text
\usepackage{comment} % For commenting out large sections of text

% Redefinitions -----------------------------------

% Redefine abstract to not include a title and to display the text in italics
\renewenvironment{abstract}
 {\small
  \begin{center}
  \vspace{-1em}\vspace{0pt}
  \end{center}
  \quotation\itshape}
 {\endquotation}

 % Tables -----------------------------------

% Modelsummary tables

\usepackage{booktabs}
\usepackage{siunitx}
\newcolumntype{d}{S[
    input-open-uncertainty=,
    input-close-uncertainty=,
    parse-numbers = false,
    table-align-text-pre=false,
    table-align-text-post=false
 ]}

\usepackage{threeparttable}

% Figures -----------------------------------

\usepackage{float}
\usepackage{graphicx}

% Packages for file structure ----------------------

\usepackage{subfiles} % Includes the \subfix command which is like here() in R

% Bibiliography -----------------------------------

% Needs to be included so the TeX file compiles on VS Code
%\usepackage[authordate, backend=biber, notitle, dateabbrev = true, short]{biblatex-chicago}
%\usepackage[reflist, style = windycity, backend = biber]{biblatex}
\usepackage[backend = biber, style = apa]{biblatex}    
    \addbibresource{\subfix{references.bib}}

\usepackage{url}
  \urlstyle{same}

% ======================== Document: Main ======================== %

\begin{document}

% Title

\maketitle

% Abstract
\begin{abstract}
    The idea that positive spillover from research efforts disincentivizes its private sector investment has led to a widespread support for fiscal incentives for innovation. While the literature on R\&D tax credits is extensive, mostly finding positive effects, the effects of alternative innovation policies are understudied. I examine the impact of the Alberta Investor Tax Credit (AITC) in Canada, which provided a tax credits to investment in innovative businesses. Exploiting variation from patent application counts, I find that the AITC had heterogeneous effects across International Patent Classification (IPC) section. Human neccessity patents increased by 0.5\% and textile patents by 0.3\%, while fixed construction patents decreased by 0.3\%, resulting in an overall null effect on total patent applications. 
\end{abstract}


%In this paper, I examine the impact of the Alberta Investor Tax Credit (AITC) in Canada, which provided a tax credits to investment in innovative businesses. I find that the AITC had an overall null effect on total patent application counts, but a positive effect on human necessity and textile patents, and a negative effect on fixed construction patents.

% The standard result from modern economic theory which states that the positive spillover from investment in research \& development will result in private underinvestment in innovation has led to a widespread support for fiscal incentives for R\&D expenditure. While the literature has mostly found postive effects of these policies, it is unclear whether fiscal incentives impact innovation. In this paper, I examine the effect of the Alberta Investor Tax Credit (AITC) on patent applications in Canadian provinces, where the tax treatment on R\&D has been recognized as one of the most generous in the world. Using two-way fixed effects difference-in-differences and event study regressions, I find that the AITC had a null overall effect on patent applications. This result challenges the conventional wisdom that fiscal incentives for R\&D expenditure are effective in promoting innovation.

\clearpage

% Sections ----------------------------------------------------------------

% Introduction

\subfile{\subfix{sections/1_introduction}}

% Institutional background 

\subfile{\subfix{sections/2_institutional_background}}

% Empirical strategy

\subfile{\subfix{sections/3_empirical_strategy}}

% Results 

\subfile{\subfix{sections/4_results}}

% Conclusion

\subfile{\subfix{sections/5_conclusion}}

% Acknowledgements

\subfile{\subfix{sections/6_acknowledgements}}

\clearpage

% References

\printbibliography

\clearpage

% Appendices ----------------------------------------------------------------

\setcounter{section}{0}

\renewcommand{\thesection}{\Alph{section}}
\renewcommand{\thetable}{\thesection.\arabic{table}}

% Appendix A: Descriptive statistics

\subfile{\subfix{sections/7_appendixA_descriptive_stats}}

% Appendix B: Difference-in-differences models

\subfile{\subfix{sections/8_appendixB_parties_models}}

% Appendix C: Robustness checks

\subfile{\subfix{sections/9_appendixC_month_panel_models}}

\end{document}