\documentclass[../main.tex]{subfiles}

% ======================== Document: Acknowledgements ======================== %

\begin{document}
\section{Acknowledgements}
\label{sec:acknowledgements}

I used R version 4.3.2 (R Core Team 2023) and the following R packages for this paper: data.table v. 1.15.0 (Barrett et al. 2024), fixest v. 0.11.2 (Bergé 2018), ggfixest v. 0.1.0 (McDermott 2023), janitor v. 2.2.0 (Firke 2023), kableExtra v. 1.4.0 (Zhu 2024), modelsummary v. 1.4.5 (Arel-Bundock 2022), rgovcan v. 1.0.4 (Lucet 2024), sandwich v. 3.1.0 (Zeileis 2004, 2006; Zeileis, Köll, and Graham 2020), scales v. 1.3.0 (Wickham, Pedersen, and
Seidel 2023), statcanR v. 0.2.6 (Warin and Le Duc 2023), and tidyverse v. 2.0.0 (Wickham et al. 2019). Replication materials are available on the project's \href{https://github.com/dsanchezp18/econ899}{GitHub repository}.

I am grateful to Natural Resources Canada Strategic Policy and Innovation Branch's Clean Technology Data Strategy team for guiding me on the Canadian data and public policy environment, which was instrumental to the completion of this project. I am eternally grateful to my MA Paper instructor, Chris Bidner, for providing invaluable feedback and confidence in my work when I needed it the most. I thank my newly found friends from Simon Fraser University for their support throughout the two most challenging years of my life.

Agradezco a Alejandra, siempre mi primera proofreader, por su paciencia y dedicación a ayudarme, con cariño, a continuar mis metas, aún cuando las razones aparentaban nublarse en mi cabeza. El esfuerzo que representó este trabajo está dedicado a mi familia, quiénes a diario hacen un esfuerzo gigantesco en lo tangible y emocional para verme ser una mejor persona en otro país. Una vez más, espero que este trabajo, con todas sus imperfecciones, alcance a ser un pequeño monumento a Jorge Pazmiño, quien junto al resto de mi familia proporciona apoyo incondicional para permitirme continuar con mi camino. ¡Gracias a todos! %

\end{document}