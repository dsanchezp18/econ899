\documentclass[../main.tex]{subfiles}

% ======================== Preamble ======================== %

% ======================== Document: Empirical Strategy ======================== %

\begin{document}
\section{Empirical Strategy}
\label{sec:empirical_strategy}

\subsection{Data}

I employ a novel administrative dataset from the Canadian Intellectual Property Office, the IP Horizons Patent Researcher Datasets. The dataset identifies patents in Canada as far back as 1869 and up to 2021 \parencite{patents_cipo_datasets}. All parties involved in a patent application can be identified along with their registered province of origin if the party comes from Canada and the state if the party is recorded in the United States.  

With these data, I construct province-quarter and province-month panels of patents from January 2001 to June 2021, mapping a patent to a province based on the province of origin of the majority parties involved in an application. This period corresponds to the modern Canadian intellectual property institutional context, which I reviewed on Section II . I only include 2021 as a partial year since the data present a downward trend for all provinces in late 2021, suggesting filings are yet to be updated for recent periods. Further, I drop Newfoundland and Labrador, Prince Edward Island, Yukon and Nunavut due to missing observations on most explanatory variables. The explained variable of interest is patent applications. To allow for heterogeneity in treatment, I separate patents by their International Patent Classification section, which defines a broad classification of the technology being patented. For robustness checks, I also consider the number of parties involved in a patent application as explained variables for robustness checks, separating by party type (inventors, owners and applicants)


\subsection{Identification Strategy}

\end{document}