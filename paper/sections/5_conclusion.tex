\documentclass[../main.tex]{subfiles}

% ======================== Document: Conclusions ======================== %

\begin{document}
\section{Conclusion}
\label{sec:conclusions}

This paper estimates the impact of an investment tax credit in Canada, the Alberta Investor Tax Credit (AITC) on innovation outcomes. The AITC provided a thirty percent tax credit to investments in small businesses that undertook research, development and commercialization activities in Alberta. Exploiting variation in patent applications from a novel administrative dataset, I use difference-in-differences and event study regression to estimate the effect of the policy. I find that the AITC had a null effect on patent applications from Albertan parties. By allowing for heterogeneity in treatment effects across International Patent Classification (IPC) sections, I find evidence that the policy increased applications for textile and human necessity inventions, yet decreased applications for fixed construction inventions. However, due to uncommon trends between Alberta and other Canadian provinces not subject to the AITC, I cannot verify whether these effects are due to the policy or unobserved factors. Hence, I provide evidence which supports a null total effect of the AITC on patent applications. These results are robust to the use of patent application parties as the explained variable, which ensure that results are not driven by my mapping of patent applications to provinces. Further, the results are validated using monthly data on patent applications. 

These results challenge the effectiveness of tax credits aimed to promote innovation. If the AITC had a positive effect on R\&D expenditure, an increase in patent applications would be expected. Data limitations do not allow me to validate the positive effect of R\&D tax credits from the literature, however, the null effect on patent applications suggests that any increased effect on R\&D expenditure did not translate to increased patent applications. A possible mechanism for this result is the relabeling of general business expenditures as R\&D in the presence of fiscal incentives \parencite{chen_etal21}. Given that the AITC required businesses to undertake R\&D activities to be eligible for AITC investment, it is possible that this relabeling occurred, which allowed businesses that were not innovative to claim the tax credit. Opportunistic behaviour by businesses to obtain AITC funding would also explain the null effect, even in the presence of non-R\&D innovation \parencite{xie_etal19}. This paper is a first step in understanding how non-traditional fiscal incentives can affect innovation outcomes. 

My paper's contribution to the literature is threefold. By using patent application counts, I focus on a short-term innovation outcome, which allows me to avoid the mentioned R\&D relabeling issue altogether. Similar work has focused on how R\&D tax credits affect R\&D expenditure, mostly finding positive effects. No work has been done on how intellectual property responds to fiscal incentives, which makes this the first result to correct for financial statement manipulation and the omission of non-R\&D innovation. Further, I analyze an investment tax credit, an understudied policy in the literature. Prior studies have typically focused on the impacts of tax credits on R\&D expenditure, giving little attention to alternative policies. To my knowledge, no other paper has focused how providing easier access to financing for innovative businesses affects innovation outcomes. While traditional tax credits have typically not had an innovation motive, the introduction of the AITC shows how governments can use tax credits to promote innovation in non-traditional ways. Finally, I extend the literature on Canadian innovation policy, which has focused on the Scientific Research and Experimental Development (SR\&ED) programs. By studying the AITC, I motivate the discussion on how other fiscal incentives at the provincial level can affect innovation outcomes.

Ultimately, while it is valuable to observe patents as an innovation outcome, these results should be validated with the traditional R\&D expenditure outcome. My inability to observe R\&D expenditure at the province level due to data limitations allows for other mechanisms that can explain the null effect, which do not relate to the ineffectiveness of the AITC intervention. One is the possibility that patent data does not capture innovation that happens outside the patent system \parencite{moser13}. It is important to validate the paper’s results by examining the effectiveness of the intervention using financial statement data from small firms. By doing this, it will be possible to determine the mechanism through which the null effect is observed. In the same line, there is evidence that patent activity does not reflect innovation but strategic responses to regulatory changes \parencite{graham_mowrey04}. Further, recent evidence shows that patenting activity has slightly decreased in Canada, in favour of trademarks and industrial designs \parencite{canadianintellectualpropertyoffice22}. To better understand the value of patent counts as measures of innovation, trademarks and industrial designs should be used as explained variables. 

Concerning threats to causal identification, while the results from my event study regressions support causal identification in the quarterly data and the total patent applications, the results are less conclusive for other outcomes. The existence of a predictable deviation from common trends between Alberta and the control provinces in the monthly data might mean that Alberta is more sensitive to seasonal shocks to innovation and research activities. Reforms in intellectual property law as well as reforms in the SR\&ED programs in any Canadian province could be causing this deviation. Further, the existence of persistent deviations from common trends across IPC sections suggests that the AITC may have had heterogeneous effects on different types of innovation, which I am unable to identify. The use of a TWFE difference-in-differences estimator may be inadequate for this type of quasi-experimental setting. Future research should focus using other methods, such as matching \parencite{caliendo_kopeinig08a} and synthetic control \parencite{abadie_etal10} for the definition of better control groups to best ensure causal identification. Using these methods will also allow for the evaluation of policies on outcomes which have are fundamentally unstable across time in the institutional context, such as employment. Further, being able to causally identify the effects of policies across different types of inventions will be able to determine if governments crowd out certain types of innovation with fiscal incentives. 

\end{document}