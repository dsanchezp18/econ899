\documentclass[../main.tex]{subfiles}

% ======================== Document: Institutional background ======================== %

\begin{document}
\section{Institutional context}
\label{sec:institutional_background}

In this section, I review the three main institutional factors which affect the empirical setting of my research design. These are the fiscal incentives of federal and provincial governments for research and development (R\&D) expenditures, the Canadian intellectual property environment and the details of the Alberta Investor Tax Credit (AITC) program.

\subsection{The Scientific Research and Experimental Development Credit}

Canada has been characterized as one of the most generous jurisdictions for R\&D credits \parencite{mckenzie08} as well as a pioneer in their design \parencite{mansfield_switzer85a}. According to the \textcite{canadarevenueagency23}, Canadian firms could deduct 100\% of current research expenditure as early as 1941\footnote{\textcite{mansfield_switzer85a} contradicts the Canada Revenue Agency's official account, stating that it was only since 1961 that the federal government effectively allowed companies to fully deduct capital and current research expenditure from federal taxable income.}. In 1962, an experimental tax incentive was created after a change to the \textit{Income Tax Act}. This program would undergo various changes over the years, taking the form of a full expenditure deduction plus a non-taxable tax credit by 1984. The current program's name - the \textit{Scientific Research and Experimental Development Credit} was given in 1986 after an amendment to the \textit{Income Tax Act}. The tax credit has been place since then, with complex rules and regulations that have been updated over the years \parencite{canadarevenueagency15}.

Canada is unique in that most provinces offer additional incentives. The Provincial SR\&ED tax credits are similar to the federal SR\&ED tax credit, but they are administered by the provinces as a top-up \parencite{warda00}. The provincial programs started being implemented in the 1980s, and by the early 2000s, most province's had adopted them \parencite{warda98,mckenzie05}. Alberta implemented the program in 2009 \parencite{brouillete13}. A combination of provincial and federal SR\&ED tax credits constitute the modern Canadian R\&D tax incentive system as of 2021.

Given the well documented effects of these policies on R\&D expenditures \parencite{mansfield_switzer85b, agrawal_etal20, becker15}, it is sensible to believe that policy efforts like these have an ongoing impact on innovation. It is crucial to consider time trends to control for federal policy changes in any type of policy evaluation of other fiscal incentives like the AITC. Further, since the provincial programs are frequently reformed and R\&D labour is eligible for credits \parencite{mckenzie05}, provincial SR\&ED tax credits pose threats to the identification of any other policy effects which targets innovation. The fact that Alberta imposed its own SR\&ED tax credit in 2009 could pose a threat to the identification of the AITC's effect on innovation in a quasi-experimental setting, which makes the use of event study regressions critical to validate results from a difference-in-differences approach.

\subsection{Intellectual property in Canada}

The Canadian intellectual property environment, as other countries, allows firms to hold innovation assets, and thus is mostly determined by laws and regulations by government agencies. The Canadian Intellectual Property Office (CIPO) is the federal agency responsible for the administration of intellectual property rights in Canada. CIPO is responsible for the registration of patents, trademarks, copyrights, industrial designs and integrated circuit topographies. Specifically, patents protect new inventions or processes. The process of applying for a patent in Canada is similar to that of other countries. If granted, a patent will be valid within Canada only, but can be owned by foreign parties. In fact, \textcite{beaudry_schiffauerova11} determine that in the nanotechnology industry in Canada, 50\% of patents are owned by foreign assignees. In the IP Horizons dataset\dots

% How many owners... foreign? 


The Canadian legal framework has recently transitioned from a system derived from British law to a system more aligned with the United States, partly in response to the North American Free Trade Agreement (NAFTA) and the United States-Mexico-Canada Agreement (USMCA), which required member countries to adopt harmonized intellectual property systems \parencite{putnam06}. Apart from these, Canada, as other countries, must adhere to a number of international treaties and agreements which govern the level of intellectual property protection in the country. Notably, a key difference between the Canadian and the American framework is that the former is much less focused on patent and copyright quality, which means that an agent is much more likely to infringe on intellectual property rights in Canada, and that patent applications in Canada are more likely to be granted than in the United States\parencite{vaver06}. 

% Patent time series

Intellectual property plays an important role in the interaction of competing industries in global markets, with recent research showing Canadian firms underperform vs. American counterparts due to worse intellectual property frameworks \parencite{carew_etal06}, as well as in international trade, with firms preferring to trade with countries with stronger intellectual property rights \parencite{rafiquzzaman02}. An accurate impact evaluation must consider the effects that foreign influences may have on the local intellectual property environment.

\end{document}