\documentclass[../main.tex]{subfiles}

% ======================== Document: Institutional background ======================== %

\begin{document}
\section{Institutional context}
\label{sec:institutional_background}

In this section, I review the three main institutional factors of my empirical setting. These are the fiscal incentives for research and development (R\&D) expenditures of Canadian federal and provincial governments, the intellectual property environment, with a specific focus on patents and the details of the Alberta Investor Tax Credit (AITC) program.

\subsection{The Scientific Research and Experimental Development Credits}

Canada has been characterized as one of the most generous jurisdictions for R\&D credits \parencite{mckenzie08} as well as a pioneer in their design \parencite{mansfield_switzer85a}. According to the \textcite{canadarevenueagency23}, Canadian firms could deduct 100\% of current research expenditures as early as 1941\footnote{\textcite{mansfield_switzer85a} contradicts the Canada Revenue Agency's official account, stating that it was only since 1961 that the federal government effectively allowed companies to fully deduct capital and current research expenditure from federal taxable income.}. In 1962, an experimental tax incentive was created after a change to the \textit{Income Tax Act}. This program would undergo various changes over the years, taking the form of a full expenditure deduction plus a tax credit by 1984. The current program's name - the \textit{Scientific Research and Experimental Development Credit} was given in 1986 after an amendment to the \textit{Income Tax Act}. The tax credit has been place since then, with complex rules and regulations that have been updated over the years \parencite{canadarevenueagency15}.

Canada is unique in that most provinces offer additional incentives. The Provincial SR\&ED tax credits are similar to the federal SR\&ED tax credit, but they are administered by the provinces as a top-up \parencite{warda00}. The provincial programs started being implemented in the 1980s, and by the early 2000s, most province's had adopted them \parencite{warda98,mckenzie05}. Alberta implemented the program in 2009 \parencite{brouillete13}.

Given the well documented effects of these policies on R\&D expenditures \parencite{mansfield_switzer85b, agrawal_etal20, becker15}, it is sensible to believe that policy efforts like these have an ongoing impact on innovation. It is crucial to consider time trends to control for federal policy changes in any type of policy evaluation of other fiscal incentives like the AITC. Further, since the provincial programs are frequently reformed \parencite{mckenzie05}, provincial SR\&ED tax credits pose threats to the identification of any other policy effects which targets innovation. The fact that Alberta imposed its own SR\&ED tax credit in 2009 could pose a threat to the identification of the AITC's effect on innovation in a quasi-experimental setting, which makes the use of event study regressions critical to validate results from a difference-in-differences approach.

\subsection{Intellectual property in Canada}

The Canadian Intellectual Property Office (CIPO) is the federal agency responsible for the administration of intellectual property rights in Canada, managing patents, trademarks and industrial designs. Patents protect innovative products, compositions or machines. The process of applying for a patent in Canada is similar to that of other countries, where inventors prepare and submit an application to the CIPO. The parties in an application can be inventors, owners\footnote{In most cases, inventors also hold legal ownership of the patent, however, see \textcite{alam_etal22, beaudry_schiffauerova11} for relevant discussions of foreign ownership of Canadian inventionships.}, agents, or other applicants\footnote{These applicants would fall under the category of legal representatives under the \textit{Patent Act} of Canada, which are heirs, executors, administrators of the estate, or any other actor who acts on behalf of the inventor in the patent application \parencite{Patent85}. This is relevant in the case of a company which hires employees to work on patent applications.}. Patent agents are external agents, commonly hired by the inventorship team, to assess the inventors on the patent application \parencite{putnam06}. After sending the application, it receives its filing date\footnote{According to \textcite{canadianintellectualpropertyoffice21}, incomplete applications will be returned to the parties for reapplication with a two month grace period. Applications which are not sent back by the parties are considered abandoned.}, and the parties have a four-year period to request an examination date, where the CIPO will evaluate the invention to grant the patent. If granted, it will be valid for 20 years only within Canada \parencite{abbes_etal22}. The patent protects the invention from being used, made, or sold by others without the inventor's permission. 

% On average, how much time does it take for a patent to get granted.

% In fact, \textcite{beaudry_schiffauerova11} determine that in the nanotechnology industry in Canada, 50\% of patents are owned by foreign assignees.

% How many owners... foreign? 

The Canadian legal intellectual property framework has recently transitioned from a system derived from British law to a system more aligned with the United States, partly in response to the North American Free Trade Agreement (NAFTA) and the United States-Mexico-Canada Agreement (USMCA), which required member countries to adopt harmonized intellectual property systems \parencite{putnam06}. Nevertheless, a key difference between the Canadian and the American framework is that the former is much less focused on patent and copyright quality. This means that intellectual property rights infringement is much more likely in Canada, and that patent applications in Canada have a simpler and faster application process relative to the United States \parencite{vaver06}. 

% Patent time series ? 

Apart from the North American trade agreements, Canada adheres to international treaties and agreements which govern the level of intellectual property protection in the country. These are the World Intellectual Property Organization (WIPO) treaties, the Paris Convention for the Protection of Industrial Property, and the Patent Cooperation Treaty (PCT). The WIPO treaties are of particular importance, as they allow to classify patents in a standardized way - the International Patent Classification (IPC). There are 8 patent classes, which represent broad categories of inventions; within each class, there are subclasses and groups\footnote{A patent may be mapped to more than one IPC section}. These are: A - Human Necessities, B - Performing Operations; Transporting, C - Chemistry; Metallurgy, D - Textiles; Paper, E - Fixed Constructions, F - Mechanical Engineering; Lighting; Heating; Weapons; Blasting, G - Physics, H - Electricity \parencite{worldintellectualpropertyorganization24}. The human necessities class is particularly broad, as it includes inventions related to food, clothing, medicine, among others. 
 
The role of international treaties also respond to the important role that intellectual property plays in the international context. For example, recent research has shown how Canadian firms underperform vs. their American counterparts due to worse intellectual property frameworks \parencite{carew_etal06}. Further, Canadian firms have been shown to be more likely to export to countries with stronger intellectual property protections \parencite{rafiquzzaman02}. An accurate impact evaluation must consider the effects that foreign influences may have on the local intellectual property environment.

\end{document}