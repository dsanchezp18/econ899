\documentclass[../main.tex]{subfiles}

% ======================== Document: Institutional background ======================== %

\begin{document}
\section{Institutional context}
\label{sec:institutional_background}

The modern Canadian institutional context which is relevant to the Alberta Investor Tax Credit (AITC) includes two main pillars: the fiscal incentives of federal and provincial governments for research and development (R\&D) expenditures and the intellectual property environment. I briefly review both in this section, and then discuss relevant details about the AITC policy.

Canada has been characterized as one of the most generous jurisdictions for R\&D credits \parencite{mckenzie08} as well as a pioneer in their design \parencite{mansfield_switzer85a}. In 1961, the federal government allowed companies to partially deduct R\&D expenditures from taxable income. The tax incentive for R\&D expenditure underwent several changes since 1962, taking the form of a full expenditure deduction plus a non-taxable tax credit by 1984. The cost of forgone revenue to the government went from C\$15 million to C\$100 million from 1962 to 1984. 

Apart from the federal incentives for R\& D incentives, Canada is unique in that most provinces offer additional incentives. The Provincial Scientific Research and Experimental Development (SR\&ED) tax credits are similar to the federal SR\&ED tax credit, but they are administered by the provinces as a top-up to the federal credit \parencite{warda00}. The provincial programs started being implemented in the 1980s, and by the early 2000's, most province's had adopted them \parencite{warda98,mckenzie05}. A combination of provincial and federal SR\&ED tax credits constitute the modern Canadian R\&D tax incentive system as of 2021.

Given the well documented effects of these policies on R\&D expenditures \parencite{becker15}, it is sensible to believe that federal policy efforts like these have had an impact on innovation in Canada, hence the importance of considering time trends in any type of policy evaluation of other fiscal incentives like the AITC. Further, since the provincial programs are frequently reformed and R\&D labour is elegible for credits \parencite{mckenzie05}, provincial SR\&ED tax credits pose threats to the identification of any other policy effects which targets innovation.

\end{document}