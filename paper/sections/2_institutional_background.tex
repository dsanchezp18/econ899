\documentclass[../main.tex]{subfiles}

% ======================== Document: Institutional background ======================== %

\begin{document}
\section{Institutional context}
\label{sec:institutional_background}

In this section, I review the details of the Alberta Investor Tax Credit (AITC) program. Further, I review the intellectual property environment in Canada, which will be directly relevant to the definition of my explained variable. Finally, I review the existing incentives for research and development (R\&D) expenditures of Canadian federal and provincial governments, which are relevant to the identification strategy of the AITC's effect on innovation.

\subsection{The Alberta Investor Tax Credit}

The AITC was a three-year program initiated by the Government of Alberta in January 2017, offering a thirty percent tax credit to \enquote{investors who provide capital to Alberta small businesses doing research, development or commercialization of new technology, new products or new processes} \parencite[p.1]{albertaeconomicdevelopmentandtrade17}\footnote{Tourism, interactive media, post-production and visual effects industries were also targeted by the program.}. The program was part of the \textit{Investing in a Diversified Economy Act}\nocite{Investing16}, which also started the Capital Investment Tax Credit (CITC)\footnote{The CITC returned the value of purchases of machinery, equipment and buildings as a tax credit. While this program may have a spillover effect on innovation through a reduced cost of innovation investment, absent the AITC, there is no reason to believe a broad capital expenditure tax credit would impact innovation products.}. Both programs were phased out in 2019, and no additional funding was given to companies after March 2020 \parencite{albertaeconomicdevelopmentandtrade19}. The AITC was communicated as a solution to Alberta's lag in \enquote{venture capital dollars} \parencite[p.1]{albertaeconomicdevelopmentandtrade17} relative to other large Canadian provinces, as well as a way to foster employment \parencite{zabjeck16}.

The program required businesses to register with the government as Venture Capital Corporations (ECC) or Eligible Business Corporations (EBC), which would then be able to raise equity capital from investors. Only investors who had paid corporate or personal taxes in the province were eligible. While the \textit{Investing in a Diversified Economy Act} was passed in January 2017, eligible investments in VCCs and ECCs were available to be claimed as credits retroactively from April 2016 onwards.

To qualify as \enquote{small}, businesses could not have more than 100 employees. Additionally, they were required to pay at least 50\%-75\% of wages to employees working in the province\footnote{Depending if the company was an exporter or non-exporter, respectively.}, have at least C\$25,000 in equity capital, and have at least 80\% of assets in the province \parencite{albertaeconomicdevelopmentandtrade19}. Businesses that were engaged in research, development and commercialization of proprietary technologies were one of the main targets of the program. These companies needed to be engaged in \enquote{the process of introducing a new product or production method and making it available to the public market. This includes the commercial production of proprietary technologies that are capable of improving the processing and manufacturing of goods and services.}\parencite[19]{albertaeconomicdevelopmentandtrade19}\footnote{The \textit{Alberta Investor Tax Credits Regulation}\nocite{Alberta19a} specifies that \enquote{companies needed to be engaged in the research, development and commercialization of proprietary technologies produced within Alberta, including services that are directly associated with the export of the technology and are provided inside or outside of Alberta}(p.9).}. Mining, financial services and agricultural activities were ineligible for receiving AITC funding.

\subsection{Intellectual property in Canada}

The Canadian Intellectual Property Office (CIPO) is the federal agency responsible for the administration of intellectual property rights in Canada, managing patents, trademarks and industrial designs. Patents protect innovative products, compositions or machines. To apply for a patent, inventors prepare and submit an application to the CIPO. 

The parties in an application can be inventors, owners\footnote{In most cases, inventors also hold legal ownership of the patent, however, see \textcite{alam_etal22, beaudry_schiffauerova11} for relevant discussions of foreign ownership of Canadian inventions.}, agents, or other applicants\footnote{These applicants would fall under the category of legal representatives under the \textit{Patent Act} of Canada, which are heirs, executors, administrators of the estate, or any other actor who acts on behalf of the inventor in the patent application \parencite{Patent85}. Multiple applicants are relevant in the case of a company that hires employees to work on patent applications.}. Patent agents are external agents, commonly hired by the inventorship team, to assess the inventors on the application \parencite{putnam06}. After sending the application, the team receives a filing date\footnote{According to the \textcite{canadianintellectualpropertyoffice21}, incomplete applications will be returned to the parties for reapplication with a two-month grace period. Applications not sent back by the parties are considered abandoned.}, and the parties have a four-year period to request an examination date when the CIPO will evaluate the invention to grant the patent. If granted, it will be valid for 20 years only within Canada \parencite{abbes_etal22}. The patent protects the invention from being used, made, or sold by others without the inventor's permission. When expired, a patent can be renewed through a renewal application. 

Canada adheres to international treaties and agreements which govern the level of intellectual property protection in the country. These are the World Intellectual Property Organization (WIPO) treaties, the Paris Convention for the Protection of Industrial Property, and the Patent Cooperation Treaty (PCT). The WIPO treaties are of particular importance, as they allow classifying patents in a standardized way: the International Patent Classification (IPC). There are 8 patent classes, which represent broad categories of inventions; within each class, there are subclasses and groups\footnote{A patent may be mapped to more than one IPC section.}. These are: A - Human Necessities, B - Performing Operations; Transporting, C - Chemistry; Metallurgy, D - Textiles; Paper, E - Fixed Constructions, F - Mechanical Engineering; Lighting; Heating; Weapons; Blasting, G - Physics, H - Electricity \parencite{worldintellectualpropertyorganization24}. The human necessities class is particularly broad, as it includes inventions related to food, clothing, medicine, among others. 
 
The role of international treaties also responds to the important role that intellectual property plays in the international context. For example, recent research has shown how Canadian firms underperform vs. their American counterparts due to worse intellectual property frameworks \parencite{carew_etal06}. Further, Canadian firms are more likely to export to countries with stronger intellectual property protections \parencite{rafiquzzaman02}. An accurate impact evaluation must consider the effects that foreign influences may have on the local intellectual property environment.

\subsection{The Scientific Research and Experimental Development Credits}

Canada has been characterized as one of the most generous jurisdictions for R\&D credits \parencite{mckenzie08} as well as a pioneer in their design \parencite{mansfield_switzer85a}. According to the \textcite{canadarevenueagency23}, Canadian firms could deduct 100\% of current research expenditures as early as 1941\footnote{\textcite{mansfield_switzer85a} contradict the Canada Revenue Agency's official account, stating that it was only since 1961 that the federal government effectively allowed companies to fully deduct capital and current research expenditure from federal taxable income.}. In 1962, an experimental tax incentive was created after a change to the \textit{Income Tax Act}. This program would undergo various changes over the years, taking the form of a full expenditure deduction plus a tax credit by 1984. The current program's name, the \textit{Scientific Research and Experimental Development Credit}, was given in 1986. The program, with complex rules and regulations that have been amended over the years \parencite{canadarevenueagency15}.

Canada is unique in that most provinces offer additional incentives. The Provincial SR\&ED tax credits are similar to the federal SR\&ED tax credit, but the provinces administer them \parencite{warda00}. The provincial programs started in the 1980s, and by the early 2000s, most provinces had adopted them \parencite{warda98,mckenzie05}. Alberta implemented the program in 2009 \parencite{brouillete13}.

Given the well-documented effects of these policies on R\&D expenditures \parencite{mansfield_switzer85b, agrawal_etal20, becker15}, it is sensible to believe that policy efforts like these have an ongoing impact on innovation. It is crucial to consider time trends to control for federal SR\&ED changes. Further, since the provincial programs are frequently reformed \parencite{mckenzie05}, provincial SR\&ED tax credits pose threats to causal identification. The fact that Alberta imposed an SR\&ED tax credit in 2009 could pose a threat to the identification of the AITC's effect on innovation in a quasi-experimental setting, which makes the use of event study regressions critical to validate results from a difference-in-differences approach.

\end{document}