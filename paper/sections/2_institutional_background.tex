\documentclass[../main.tex]{subfiles}

% ======================== Document: Institutional background ======================== %

\begin{document}
\section{Institutional context}
\label{sec:institutional_background}

The modern Canadian institutional context which is relevant to the Alberta Investor Tax Credit (AITC) includes two main pillars: the fiscal incentives of federal and provincial governments for research and development (R\&D) expenditures by firms and individuals, and the legal intellectual property environment. I briefly review both in this section, and then discuss relevant details about the AITC policy.

Canada has been characterized as one of the most generous jurisdictions for R\&D credits \parencite{mckenzie08} as well as a pioneer in their design \parencite{mansfield_switzer85a}. In 1961, the federal government allowed companies to partially deduct R\&D expenditures from taxable income. The tax incentive for R\&D expenditure underwent several changes since 1962, taking the form of a full expenditure deduction plus a non-taxable tax credit by 1984. The cost of forgone revenue to the government went from C\$15 million to C\$100 million from 1962 to 1984. 

\end{document}