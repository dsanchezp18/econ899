\documentclass[../main.tex]{subfiles}

% ======================== Document: Institutional background ======================== %

\begin{document}
\section{Institutional context}
\label{sec:institutional_background}

Given that the Alberta Investor Tax Credit (AITC) aimed to foster innovation within Alberta, the relevant Canadian institutional context includes two main pillars: the fiscal incentives of federal and provincial governments for research and development (R\&D) expenditures and the intellectual property environment. In this section, I review both of these and then discuss relevant details about the AITC.

\subsection{The Scientific Research and Experimental Development Credit}

Canada has been characterized as one of the most generous jurisdictions for R\&D credits \parencite{mckenzie08} as well as a pioneer in their design \parencite{mansfield_switzer85a}. According to the \textcite{canadarevenueagency23}, Canadian firms could deduct 100\% of current research expenditure as early as 1941\footnote{\textcite{mansfield_switzer85a} contradicts the Canada Revenue Agency's official account, stating that it was only since 1961 that the federal government effectively allowed companies to fully deduct capital and current research expenditure from federal taxable income.}. In 1962, an experimental tax incentive was created after a change to the \textit{Income Tax Act}. This program would undergo various changes over the years, taking the form of a full expenditure deduction plus a non-taxable tax credit by 1984. The current program's name - the \textit{Scientific Research and Experimental Development Credit} was given in 1986 after an amendment to the \textit{Income Tax Act}. The tax credit has been place since then, with complex rules and regulations that have been updated over the years \parencite{canadarevenueagency15}.

Canada is unique in that most provinces offer additional incentives. The Provincial SR\&ED tax credits are similar to the federal SR\&ED tax credit, but they are administered by the provinces as a top-up \parencite{warda00}. The provincial programs started being implemented in the 1980s, and by the early 2000s, most province's had adopted them \parencite{warda98,mckenzie05}. Alberta implemented the program in 2009 \parencite{brouillete13}. A combination of provincial and federal SR\&ED tax credits constitute the modern Canadian R\&D tax incentive system as of 2021.

Given the well documented effects of these policies on R\&D expenditures \parencite{mansfield_switzer85b, agrawal_etal20, becker15}, it is sensible to believe that policy efforts like these have an ongoing impact on innovation. It is crucial to consider time trends to control for federal policy changes in any type of policy evaluation of other fiscal incentives like the AITC. Further, since the provincial programs are frequently reformed and R\&D labour is eligible for credits \parencite{mckenzie05}, provincial SR\&ED tax credits pose threats to the identification of any other policy effects which targets innovation. The fact that Alberta imposed its own SR\&ED tax credit in 2009 could pose a threat to the identification of the AITC's effect on innovation in a quasi-experimental setting, which makes the use of event study regressions critical to validate results from a difference-in-differences approach.

\subsection{Intellectual property in Canada}

\end{document}