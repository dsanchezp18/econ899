\documentclass[../main.tex]{subfiles}

% ======================== Document: Introduction ======================== %

\begin{document}
\section{Introduction}
\label{sec:introduction}

\textit{Creative destruction}, the process through which the invention of new products, methods or processes lead to the obsolescence of old ones, was seen by Joseph Schumpeter as the \enquote{essential fact of capitalism} \parencite[p.24]{caballero10}, and widely incorporated in economic theory as a determinant of long-run economic growth \parencite{aghion_howitt92, artz_etal10, jones95}. For a society to engage in creative destruction, it must invest in its innovative capacity, and researchers have recognized that such investment resembles resembles a public good, which results in free-riding and a suboptimal level of private innovation \parencite{bloom_etal19}. An extensive literature has emerged to study fiscal incentives for innovation, notably on research and expenditure (R\&D) tax credits, yet alternative fiscal policy has received little attention. In this paper, I investigate a fiscal policy aimed to increase investment in innovative firms, the Alberta Investment Tax Credit (AITC), and its effect on patent applications. 

The extensive literature on the effects of R\&D tax credits has mostly found that these programs positively impact R\&D expenditure\footnote{See \textcite{becker15}, \textcite{hall_etal10} and \textcite{hall_vanreenen00} for a review.} and it has been observed that countries with higher R\&D to GDP ratios grow faster \parencite{jones16}. R\&D tax credits typically involve subsidizing R\&D by lowering the firm's tax bill, hence increasing incentives for such expenses. However, observed stagnation in productivity growth in developed economies despite the growth in R\&D expenditure has questioned the validity of the R\&D and innovation relationship. Particularly, despite Canada being one of the most generous tax jurisdictions for R\&D, the country has not seen a significant R\&D intensity ratio \parencite{mckenzie06}. This apparent paradox underscores the importance of understanding how alternative fiscal incentives, such as tax credits for investments in inventive firms, can affect innovation outcomes.

In January 2017, the Government of Alberta passed the \textit{Investing in a Diversified Alberta Economy Act}, which introduced the Alberta Investor Tax Credit (AITC), a tax credit for investors who financed Albertan firms undertaking research, development and commercialization activities. As an investment tax credit, the AITC is fundamentally different to the R\&D tax credit in that it does not aim to subsidize R\&D expenditure, but rather to provide easier access to financing for innovative firms. I estimate the impact of the AITC on patent applications using a two-way fixed effects difference-in-differences design, validating causal identification through event study specifications. I map patent applications to provinces using a novel administrative dataset from the Canadian Intellectual Property Office (CIPO), using the reported locations of parties from patent applications to assign them to provinces. My results show that the AITC had an overall null effect on Alberta patent applications. Separating by International Patent Classification (IPC) sections, I find that the intervention increased human necessity and textile invention patents, yet the effect was offset by a decrease in fixed construction patents. These results are validated replicating the analyses using parties in applications as the innovation outcome and higher frequency data, ensuring that results are not driven by the mapping of applications to provinces or the aggregation of data to the quarterly frequency. The relatively small size of the standard errors in my estimates also shows that it is unlikely that the null effect is due to imprecision.

Focusing on patent applications is a useful way to measure innovation when R\&D is not observed by the researcher. Patent statistics have been extensively used to measure innovation, as they proxy the outputs of the inventive process \parencite{nordhaus69, pavitt85,trajtenberg90,artz_etal10}. Patents have been used to estimate the knowledge spillover generated by the innovation process, especifically using patent citations in patent applications \parencite{trajtenberg90,jaffe_etal93}. While using patent data has been shown to have limitations \parencite{lanjouw_etal98a}, others have been shown that patents move together with other innovation products \parencite{lanjouw_schankerman04}.

My findings make three main contributions to the literature. First, I provide the first evidence on how investment tax credits affect innovation outcomes. The AITC is a unique policy in that it provides easier access to financing for innovative firms, rather than subsidizing R\&D expenditure. The investment tax literature has typically focused on macroeconomic impacts and firm outcomes, mostly finding positive effects \parencite{pereira94, lyon89, slattery_zidar20}. This owes to the fact that invest tax credits are typically used to stimulate investment in capital goods, which has a fundamentally different purpose to the traditional R\&D tax credits. However, modern programs such as the AITC have received significant attention from the public \parencite{albertachamberofcommerce23,zabjeck16}, which may lead to the increased use of similar policy by governments. The literature should incorporate nontraditional fiscal incentives to understand how they affect innovation outcomes.

Second, to my knowledge, no work has been done in evaluating any type of fiscal incentive on intellectual property outcomes. The literature has mainly focused on how R\&D tax credits affect R\&D expenditure. Focusing on R\&D expenditure as the main measure for innovation has been criticized for not capturing innovation outside the R\&D process \parencite{xie_etal19}. Using patent applications as the innovation outcome allows me to capture innovation which is not necessarily produced through a research and development process. Further, recent evidence has shown that tax credits may induce the relabeling of non-innovative business expenditure as R\&D by firms to leverage tax savings \parencite{chen_etal21}. The incompatibility between the slowdown in productivity growth and the positive effects of R\&D tax credits on R\&D expenditure may be explained by this phenomenon. Exploiting variation in patent applications allows me to avoid this issue altogether.

Third, I extend the literature on Canadian fiscal policy, which has given centered on the Scientific Research and Experimental Development (SR\&ED) programs. The SR\&ED programs have motivated a large literature, most of which has found positive effects of the programs on R\&D expenditure \parencite{agrawal_etal20,czarnitzki_etal11,berube_mohnen09,mansfield_switzer85a,bernstein86b}. This literature has relied on firm-level outcomes and quasi-experimental designs, exploiting changes in program eligibility or provincial policy changes to estimate average treatment effects. However, these findings are also unable to reconcile the stagnation in R\&D intensity in Canada with the positive effects of the SR\&ED programs. Providing new evidence on how fiscal incentives affect innovation outcomes in Canada is crucial to understanding the Canadian innovation landscape. 

The rest of the paper proceeds as follows. Section \ref{sec:institutional_background} provides an overview of the AITC and the Canadian institutional context. Section \ref{sec:empirical_strategy} describes the empirical strategy. Section \ref{sec:results} presents the paper's results. Section \ref{sec:conclusions} concludes.

\end{document}