\documentclass[../main.tex]{subfiles}

% ======================== Document: Results ======================== %

\begin{document}
\section{Results}
\label{sec:results}

\subsection{Patent applications}

Table \ref{tab:dd_twfe_patents} presents results the estimation of Equation \ref{eq:dd_model} using patent applications as the explained variable with the province-quarter data. Specification (1) includes a baseline result with no control variables. Specification (2) includes economic controls included to account for factors which may affect the comparability of the treatment and control groups regarding firm activity and overall economic trends which vary across time and provinces. The number of foreign parties in all the province's patent applications is also included, to control for effects that foreign interested parties (particularly U.S.) can have as strategic actors for patent applications. Specification (3) considers additional controls, which are included in case that the previous ones did not account for differences in trends due to reasons other than economy, or that economic activity is not well captured by standard economic variables in Specification (2). 

The DD estimate for the effect of the AITC intervention is the coefficient on Treatment \texttimes Post, showing that the intervention led to an -6.1\% to +2.3\% change in Albertan patent applications. The baseline specification shows a negative effect, while the other two specifications show positive effects. However, these are not statistically distinguishable from zero. Standard errors for this coefficient on all three specifications are small compared to those of the controls, showing that $\hat{\beta}$ is estimated with a fairly good level of precision. This implies that it is the small magnitude of $\hat{\beta}$ which drives the low $p$-value of the hypothesis test, leading to the preliminary conclusion that the AITC intervention had no effect on innovation in the studied period. 

\begin{table}[htbp!]
    \centering
\begin{threeparttable}
    \caption{Difference-in-differences specifications for quarterly patent applications}
    \label{tab:dd_twfe_patents}
    
\begin{tabular}[t]{lccc}
\toprule
  & (1) & (2) & (3)\\
\midrule
Treatment x Post & \num{-0.061} & \num{0.028} & \num{0.023}\\
\textbf{} & \textbf{(\num{0.043})} & \textbf{(\num{0.065})} & \textbf{(\num{0.074})}\\
Ln Full-time employment &  & \num{0.825} & \num{1.104}\\
 &  & (\num{0.651}) & (\num{0.664})\\
Ln Median wage &  & \num{1.200}** & \num{1.078}**\\
 &  & (\num{0.378}) & (\num{0.432})\\
CPI &  & \num{-0.015}** & \num{-0.007}\\
 &  & (\num{0.005}) & (\num{0.008})\\
Ln +1 Business insolvencies &  & \num{-0.065}** & \num{-0.052}*\\
 &  & (\num{0.027}) & (\num{0.023})\\
Ln Intl. exports &  & \num{-0.087} & \num{-0.089}\\
 &  & (\num{0.093}) & (\num{0.119})\\
Ln Intl. imports &  & \num{0.018} & \num{0.023}\\
 &  & (\num{0.125}) & (\num{0.125})\\
Ln Retail sales &  & \num{-0.272} & \num{0.084}\\
 &  & (\num{0.421}) & (\num{0.495})\\
Ln Wholesale sales &  & \num{-0.139} & \num{-0.222}\\
 &  & (\num{0.164}) & (\num{0.150})\\
Ln Manufacturing sales &  & \num{0.276} & \num{0.216}\\
 &  & (\num{0.150}) & (\num{0.140})\\
Ln +1 Foreign patent parties &  & \num{0.142}*** & \num{0.136}***\\
 &  & (\num{0.015}) & (\num{0.016})\\
Ln International travellers &  &  & \num{-0.129}***\\
 &  &  & (\num{0.034})\\
Ln Arriving vehicles &  &  & \num{0.007}\\
 &  &  & (\num{0.005})\\
Ln Electric power generation &  &  & \num{0.070}\\
 &  &  & (\num{0.118})\\
Ln Average actual hours &  &  & \num{0.109}\\
 &  &  & (\num{0.275})\\
New housing price index &  &  & \num{-0.003}\\
 &  &  & (\num{0.002})\\
Ln Food services receipts &  &  & \num{-0.067}\\
 &  &  & (\num{0.205})\\
Ln Average job tenure &  &  & \num{-0.448}\\
 &  &  & (\num{0.372})\\
\midrule
Explained variable &  & $\ln(\text{Patents}+1)$ & \\
$N$ & \num{656} & \num{656} & \num{656}\\
Adj. $R^2$ & \num{0.975} & \num{0.980} & \num{0.980}\\
Adj. within $R^2$ & \num{0.000} & \num{0.205} & \num{0.210}\\
RMSE & \num{0.206} & \num{0.182} & \num{0.180}\\
\bottomrule
\end{tabular}
}
    \begin{tablenotes}
        \small
        \item \textit{Notes}: Clustered standard errors at the province and quarter level shown in parentheses. All specifications include fixed effects for provinces and quarters. ***$p<0.01$, **$p<0.05$, *$p<0.1$.
    \end{tablenotes}
\end{threeparttable}
\end{table}

I display the results of the event study regressions in Figure \ref{fig:event_study_patents}, which plots the $\hat{\beta}_t$ interaction coefficients in \ref{eq:event_study} with the same controls as the specifications in Table \ref{tab:dd_twfe_patents}. These results show that when controlling for time and province-varying factors, there is not a statistically significant difference between Albertan and control provinces patent applications before the intervention. This supports the key identifying assumption of the DD design. The baseline model shows several pre-policy periods where the treatment and control groups diverge, underscoring the importance of including controls in the model. However, in all specifications 2015Q4 presents a statistically significant difference in patent applications between the treatment and control groups. This can be due to random noise or to a temporary real effect, however, since it is only one period which is significant, it does not greatly threaten causal identification of the AITC effect. 

\begin{figure}[htbp!]
    \centering
    \caption{Event study plot for quarterly patent applications}
    \label{fig:event_study_patents}
    \includegraphics{\subfix{../../figures/event-studies/quarterly/patents_faceted.png}}
    \begin{minipage}{0.9\textwidth}
        \footnotesize
        \textit{Notes}: The figure shows the estimated coefficients of the interaction term between period and treatment binary variables in Equation \ref{eq:event_study} for each quarter. The points represent the point estimate, while the error bars represent the 95\% confidence cluster-robust interval. The vertical line represents the start of the AITC intervention (first expense eligibility date) in April 2016, with the reference level being the quarter before the intervention. Baseline, economic, and additional controls specifications include the controls seen in specifications (1) through (3) in Table \ref{tab:dd_twfe_patents}. 
    \end{minipage}
\end{figure}

% Note: need to add a row with the number of identified patent applications to make the table more informative.

Regarding the effect of the policy itself, results point toward an overall null effect. While there is a small positive effect in 2016Q4, it is unlikely that this is due to the policy, as the effect is not present in the following quarters. There is also no evidence of a negative effect of the policy on patent applications, as preliminarily shown by the time series plot in Figure \ref{fig:quarterly_common_trends}.

\subsection{Patents by IPC section}

In Table \ref{tab:dd_twfe_patents_by_section}, I present the results of estimating Equation \ref{eq:dd_model} allowing for heterogeneity by IPC patent section and including the controls of Specification (3) in Table \ref{tab:dd_twfe_patents}. The results show that the AITC intervention had a null effect on most of the IPC sections except A and E, corresponding to human necessities and fixed constructions. The effect on section A is positive while on section E it is negative. The two coefficients are of similar magnitude (between 42.7\% to 53.0\% in absolute value), thus justifying the null effect on total patent applications. 
   
\begin{table}[htbp!]
    \centering
\begin{threeparttable}
    \label{tab:dd_twfe_patents_by_section}
    \caption{Difference-in-differences results for quarterly patent applications by IPC section}
    
\begin{tabular}[t]{lcccccccc}
\toprule
  & (1) & (2) & (3) & (4) & (5) & (6) & (7) & (8)\\
\midrule
Treated & \num{0.409}*** & \num{0.329} & \num{0.068} & \num{0.334}** & \num{-0.577}*** & \num{0.164} & \num{-0.138} & \num{0.229}\\
 & (\num{0.077}) & (\num{0.215}) & (\num{0.186}) & (\num{0.138}) & (\num{0.062}) & (\num{0.107}) & (\num{0.193}) & (\num{0.149})\\
\midrule
IPC & A & B & C & D & E & F & G & H\\
$N$ & \num{656} & \num{656} & \num{656} & \num{656} & \num{656} & \num{656} & \num{656} & \num{656}\\
Adj. $R^2$ & \num{0.913} & \num{0.911} & \num{0.879} & \num{0.355} & \num{0.915} & \num{0.875} & \num{0.910} & \num{0.908}\\
Adj. within $R^2$ & \num{0.109} & \num{0.054} & \num{0.082} & \num{0.037} & \num{0.064} & \num{0.021} & \num{0.061} & \num{0.064}\\
RMSE & \num{0.324} & \num{0.355} & \num{0.381} & \num{0.355} & \num{0.360} & \num{0.394} & \num{0.395} & \num{0.409}\\
\bottomrule
\end{tabular}
}
    \begin{tablenotes}
        \footnotesize
        \item \textit{Notes}: Sections of the IPC are A: Human Necessities, B: Performing Operations; Transporting, C: Chemistry; Metallurgy, D: Textiles; Paper, E: Fixed Constructions, F: Mechanical Engineering; G: Physics, H: Electricity. Patents with multiple sections are not included. All specifications include controls in Specification (3) of Table \ref{tab:dd_twfe_patents}, not shown for brevity and fixed effects for provinces and quarters. Clustered standard errors at the province and quarter level shown in parentheses. ***$p<0.01$, **$p<0.05$, *$p<0.1$.
    \end{tablenotes}
\end{threeparttable}
\end{table}

% Note: need a row with the number of identified patent applications by section to make the table more informative.
% Should also try to make the table less wide so that it fits within the page.

Given that there is a smaller number of patents per IPC section for all provinces in every period, I am underpowered to detect small effects on other sections. The event study regressions in Figure \ref{fig:event_study_patents_section} provide additional insight about the intervention's effect on patent applications by IPC section. The figure displays the same coefficients and confidence intervals as those of in Figure \ref{fig:event_study_patents} now separating by IPC section and restricting to the additional controls specification.

\begin{figure}[htbp!]
    \centering
    \caption{Event study plot for quarterly patent applications by IPC section}
    \label{fig:event_study_patents_section}
    \includegraphics{\subfix{../../figures/event-studies/quarterly/patent_sections_faceted.png}}
    \begin{minipage}{0.9\textwidth}
        \footnotesize
        \textit{Notes}: The figure shows the estimated coefficients of the interaction term between period and treatment binary variables in Equation \ref{eq:event_study} for each quarter, separating by IPC section. The points represent the point estimate, while the error bars represent the 95\% confidence cluster-robust interval. The vertical line represents the start of the AITC intervention (first expense eligibility date) in April 2016, with the reference level being the quarter before the intervention. Controls are the same as those in Specification (3) in Table \ref{tab:dd_twfe_patents}. 
    \end{minipage}
\end{figure}

% include patent section names in event study plots

% need to include definition of human necessity

Human necessity (A) patents, which include agriculture, medicine and apparel related inventions, show only one significant deviation from the pre-policy trend, five quarters after the intervention (2017Q3). These patents increased by approximately 79\% relative to 2016Q1. However, pre-intervention coefficients are less stable for this section, potentially due to its broad definition.

Fixed constructions (E) patents, which include patents related to buildings, roads, and bridges, do show a significant decrease in most quarters after the intervention. Pre-policy coefficients are stable, and the effect is present in the first quarter after the intervention. This suggests that the AITC may indeed have had a negative effect on this type of patent applications. Because the policy did not particularly target this type of innovation, section E inventions may have been crowded out by other types of products which were more incentivized by the AITC.

Other notable results which were not picked up by the DD specification are the Section D patents, which include patents related to textiles and paper. While less stable than other sections in the post-policy trend, these patents show the most important increases, the highest being 238\% more patent applications 19 quarters after the intervention (2020Q3). The pre-policy trend is mostly common across the treatment and control groups (with the exception of 2015Q4, same as in the total patent event study). The DD may not capture this effect due to increases being present next to quarters without increases. 

\subsection{Robustness checks}

Appendix \ref{sec:appendixb} presents the results of the DD specifications and event studies for number of parties in patent applications as explained variables. I consider the total amount of type of parties in patent applications and also separate by specific types: inventors, owners and applicants. 

Table \ref{tab:dd_twfe_patents} shows the results of the DD estimation for all four explained variables. All specifications show a null effect, except the inventors specification, which shows a significant positive effect. Event study plots show that the effect is null for most quarters in post-policy periods for total parties, inventors and applicants, however, pre-policy trends are much less stable for these variables than for owners. Notably, there are negative effects in most pre-policy interaction terms for inventor parties, which may be driving the positive effect in the post-policy period. The positive difference for 2015Q4 is present on all four explained variables, which shows that this difference is not due to the mapping of patents to parties.

The owners specification does show an increasing trend in the last four quarters of the data, and very stable pre-policy trends. This suggests that the AITC may have had a positive effect on the number of owners in patent applications, which may be due to the policy incentivizing firms to apply for patents. Potentially, this effect may be driven by the increases in section A and D patents, as seen in the previous subsections. On all but the last quarter, the interaction terms for applicants shows a negative effect, which may explain the overall null effect of the policy on total parties.

Overall, these results present similar results to the patent applications results. This suggests that my mapping of patents to provinces is not driving the effect to zero, which could be a potential concern, especially due to the elimination of some patents which presented an equal number of parties from different provinces.

Appendix \ref{sec:appendixc} reproduces the DD specifications and event study regression using the province-month panel, constructed as described in Section \ref{sec:empirical_strategy}. DD specifications also show a null effect of the AITC intervention on the log of patent applications, considering all three specifications in Table \ref{tab:dd_twfe_patents} (baseline, economic and additional controls). The additional precision does not change my results, which further disproves the possibility of an underpowered statistical analysis driving the null effect. The event study plot shows a similar post-intervention pattern, showing a small positive effect shortly after the first month of elegibility, which is not present in the following months. The finer granularity of the data allows to identify small positive effects in the last months of 2020, which disappear in 2021. The pre-policy trend is less stable than the quarter panel, notably around 2004. Results are similar for the IPC section disaggregations, yet with considerably more instability in the pre-policy trend. 

\end{document}